\documentclass[11pt,a4paper]{report}
\usepackage[margin=1in]{geometry}
\usepackage[british,english]{babel}
\usepackage{titlesec} 
\usepackage{color}
\usepackage[export]{adjustbox}
\usepackage[hidelinks]{hyperref}
\usepackage{mathtools}
\usepackage{caption}
\usepackage{listings}
\DeclareCaptionFont{white}{ \color{white} }
\DeclareCaptionFormat{listing}{
  \colorbox[cmyk]{0.43, 0.35, 0.35,0.01 }{
    \parbox{\textwidth}{\hspace{15pt}#1#2#3}
  }
}

\titleformat{\chapter}[block]
  {\normalfont\huge\bfseries}{\thechapter.}{1em}{\Huge}
\titlespacing*{\chapter}{0pt}{-20pt}{15pt}
%Gummi|065|=)
\title{\huge \textbf{Information Retrieval Engine}\\[0.2ex]
\large Assignment 1 - Report\\[1ex]
\large Information Retrieval - 2016/2017}
\author{\large Andr\'e Lopes - 67833\\[1ex]
        Raquel Rocha - 62196}

\date{\large October 5, 2016}
\usepackage{graphicx}
\definecolor{codegreen}{rgb}{0,0.6,0}
\definecolor{codegray}{rgb}{0.5,0.5,0.5}
\definecolor{codepurple}{rgb}{0.58,0,0.82}
\definecolor{backcolour}{rgb}{0.95,0.95,0.92}
 
\lstdefinestyle{mystyle}{
    backgroundcolor=\color{backcolour},   
    commentstyle=\color{codegreen},
    keywordstyle=\color{magenta},
    numberstyle=\scriptsize\color{codegray},
    stringstyle=\color{codepurple},
    basicstyle=\footnotesize,
    breakatwhitespace=false,         
    breaklines=true,                 
    captionpos=b,                    
    keepspaces=true,                     
    showspaces=false,                
    showstringspaces=false,
    showtabs=false,     
    numbers=left,
  	numbersep=3pt,
    tabsize=2,
	framexleftmargin=10pt,
	framexrightmargin=6pt,
	escapeinside={(*@}{@*)} 
}
 
\lstset{style=mystyle}
\begin{document}
\makeatletter
    \begin{titlepage}
        	\vspace*{\fill}
        \begin{center}
            {\@title }\\[2ex] 
            {\@author}\\[2ex] 
            {\@date}\\[5ex]
			\includegraphics[scale=0.8]{deti_logo.png}
        \end{center}
        
        	\vspace*{\fill}
    \end{titlepage}
\makeatother
\hypersetup{
    colorlinks,bookmarks=true,linktoc=all
}

\renewcommand*\contentsname{Table of Contents}
%\renewcommand\lstlistlistingname{List of Code}
\tableofcontents
\vspace{2cm}
%\begingroup
%\let\clearpage\relax
%\lstlistoflistings
%\endgroup
\setcounter{secnumdepth}{2}
\chapter{Introduction}

Information retrieval is the act of collecting information from various sources, which main principle is querying the information gathered. A big example of information retrieval is Google, where a user inputs a set of keywords and gets the respective results ranked by relevance - this is called an information retrieval search engine, which is the objective of this assignment.\\\\
This report explains the structure of this assignment, in terms of class estructure and data flow of the engine. 

\vspace{2cm}
{\let\clearpage\relax \chapter{Modelling}}
%\chapter{Modelling}
In order to model the structure of the search engine, there was a need to better understand each component:
\begin{itemize}
\item Document Processor
\begin{itemize}
\item[\textperiodcentered]Opens each document to collect data (in Corpus Reader)
\end{itemize}
\item Corpus Reader
\begin{itemize}
\item[\textperiodcentered]Reads the data in each document processed by the Document Processor, according to its format
\end{itemize}
\item Tokenizer
\begin{itemize}
\item[\textperiodcentered]Divides the data collected in terms, removing stopwords, stemming and other text transformations needed
\end{itemize}
\item Indexer
\begin{itemize}
\item[\textperiodcentered]Indexes each term with a reference to the documents it belongs (postings lists)
\end{itemize}
\item Searcher
\begin{itemize}
\item[\textperiodcentered]Retrieves the documents that contain the queries token from the index
\end{itemize}
\end{itemize}
\pagebreak
\section{Class Diagram}
INSERT PICTURE HERE
\subsection{Data Flow}
Explicacao do class diagram (como é q os diferentes modulos funcionam)


\end{document}
